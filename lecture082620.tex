\documentclass[12pt]{article}
\usepackage[%
	letterpaper,%
	includeheadfoot,%
	margin=0.5in,%
	headheight=0.21in,%
	footskip=0.51in%
]{geometry}

% Heading information
\newcommand{\documentname}{Lecture Notes}
\newcommand{\classname}{Math 554 --- H01}
\newcommand{\semester}{Fall 2020}
\newcommand{\lastrevised}{\today}
\newcommand{\duedate}{August 26, 2020}
\newcommand{\yourname}{Andrew Smith}

% standard AMS math packages
\usepackage{amssymb,amsmath,amsthm}

% problem + solution Environments
\usepackage{environ}
\usepackage{tabto}
\usepackage[svgnames]{xcolor}
\usepackage{framed}
\usepackage{changepage}
\definecolor{shadecolor}{rgb}{0.8,1.00,0.95}
% the solutions are printed if the following is 1
% they are not printed if it is 0
\def\printsolutions{1}
% problem environment
\newcounter{ProblemNum}
\NewEnviron{problem}[1][]{%
\refstepcounter{ProblemNum}
\if\printsolutions1
\begin{shaded}
\else
\bigskip
\fi
\noindent
{\bf Problem%\theProblemNum%
}
\if\printsolutions0
\ifx&#1&\else (#1 points) \fi
\fi
\BODY
\if\printsolutions1
\end{shaded}
\fi
}
% subproblem environment
\newcounter{SubProblemNum}[ProblemNum]
\NewEnviron{subproblem}[1][]{%
\refstepcounter{SubProblemNum}
\if\printsolutions1
\begin{shaded}
\fi
\begin{adjustwidth}{2em}{}
\noindent
%\textbf{(\alph{SubProblemNum})}
\if\printsolutions0
\ifx&#1&\else (#1 points) \fi
\fi
\BODY
\end{adjustwidth}
\if\printsolutions1
\end{shaded}
\fi
}
% solution environment
\NewEnviron{solution}{%
\if\printsolutions1
\begin{oframed}
\noindent
\BODY
\end{oframed}
\fi
}

% Header and Footer Information
\usepackage{fancyhdr}
\usepackage{lastpage}
\pagestyle{fancy}
\lhead{\classname}
\chead{\documentname \if\printsolutions1 --- \yourname \fi}
\rhead{\semester}
\lfoot{\if\printsolutions1 \duedate \else Last Revised: \lastrevised\fi}
\cfoot{}
\rfoot{\thepage\ of \pageref{LastPage} }
\renewcommand{\headrulewidth}{0.4pt}
\renewcommand{\footrulewidth}{0.4pt}


% standard commands
\newcommand{\N}{\mathbb{N}}
\newcommand{\Z}{\mathbb{Z}}
\newcommand{\Q}{\mathbb{Q}}
\newcommand{\R}{\mathbb{R}}
\newcommand{\C}{\mathbb{C}}

\begin{document}

\if\printsolutions1
\begin{center}
\textbf{\Large \yourname}
\end{center}
\else
\begin{center}
\textbf{\Large Due: \duedate}
\end{center}
\fi

%%%%%%%%%%%%%%%%%%%%%%%%%%%%%%%%%%%%%%%%%%%%%%%%%%%%%%%%%%%%%%%%%%%%%%%%%%%%
\begin{solution}\newline
\underline{Discussion on HW1}\newline
$A\subset B\iff A\cup B=B$\newline
($\Rightarrow$) Assume $A\subset B$ to prove $A\cup B=B$\newline
Note first $B\subset A\cup B$ by definition of union.\newline
It remains to prove $A\cup B\subset B$\newline
Let $x\in A\cup B$. Then $x\in A\lor x\in B$. If $x\in B$, then done.\newline
Assume $x\in A$. Then $A\subset B\implies x\in B$.\newline
Either way $x\in B$, so $A\cup B\subset B$.\newline
($\Leftarrow$) Assume $A\cup B= B$. To prove $A\subset B$, $x\in A\implies x\in A\cup B\implies x\in B$.\newline
\underline{Recall:}\newline
\begin{center}
    $\N\subset\Z$ where \newline
    $\N = \{1,2,\dots\}\newline
    \Z=\{0\pm n:n\in \N\}\newline$
\end{center}\newline
NOTE: ($\Z,+,x$) is a ring.
    $x^{-1}$ does not \underline{necessarily} exist.\newline
    In \textit{analysis}, we need \underline{ORDERING}\newline
    $n\leq m \iff m-n\geq0$\newline
    \underline{Constructing $\Q$}\newline
    Some people would say\newline
    $\Q:=\{\frac{m}{n}:m,n\in\Z,n\neq0\}$\newline
    \underline{but} we would need equivalence\newline
    i.e. $\frac{2}{1}=\frac{4}{2}$.\newline
    \underline{Models of $\Q$ historically:}\newline
    Think of how we can describe $\sqrt{2}$ in terms of rationals.\newline
    NOTE: FieldOfFractions($\Z=\Q$).\newline
    1. \underline{CANTOR}\newline
    Identify $\sqrt{2}$ as a limit of a sequence of rational numbers. Take the equivalence class of all sequences of rational numbers converging to $\sqrt{2}$. Slight modification for arbitrary real numbers use equivalence classes of Cauchy sequences of rational numbers.\newline
    2. \underline{Dedekind} used "cuts"\newline
    We identify $\sqrt{2}$ with ($A,B$) where $A=\{x\in\Q,x^2<2\},B=\{x\in\Q, x^2>2\}$\newline
    Then $\Q=A\cup B,A\cap B=\varnothing$.\newline
    If $x\in A,y\in B$, then $x<y$. Now $\R$ is all the cuts.\newline\newline
I) \underline{Field Axioms}\newline
A1:For any $a,b\in\R$ there is a number $a+b\in\R$ and $a+b=b+a$.\newline
A2: Addition is associative\newline
A3: $\exists 0\in\R, (a+0=0+a=a)$\newline
A4: $\exists-a\forall a\in\R(a+(-a)\0)$\newline
M1: $a,b\in\R\implies ab\in\R,ab=ba$\newline
M2: multiplication is associative\newline
M3: multiplicative identity\newline
M4:$\forall a\neq 0, a^{-1} \text{ exists } a\cdot a^{-1}=a^{-1}\cdot a=1$\newline
M5: distributive\newline
II) \underline{Order Structure}($\R$ is an ordered field)\newline
O1: \underline{total} $a,b\in\R$ then exactly one of $a<b,a=b,b<a$ is true.\newline
O2: transitive\newline
O3: $a<b\implies a+c<b+c$\newline
O4: $a<b,c>0\implies ca<cb$\newline
A+M+O say $\R$ is ordered field, but $\Q$ are also.\newline
To distinguish $\R$ from $\Q$ we need completeness axiom.
\end{solution}
%%%%%%%%%%%%%%%%%%%%%%%%%%%%%%%%%%%%%%%%%%%%%%%%%%%%%%%%%%%%%%%%%%%%%%%%%%%%

\end{document}