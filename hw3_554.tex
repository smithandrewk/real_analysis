\documentclass[12pt]{article}
\usepackage[%
	letterpaper,%
	includeheadfoot,%
	margin=0.5in,%
	headheight=0.21in,%
	footskip=0.51in%
]{geometry}

% Heading information
\newcommand{\documentname}{Homework 3}
\newcommand{\classname}{Math 554 --- H01}
\newcommand{\semester}{Fall 2020}
\newcommand{\lastrevised}{\today}
\newcommand{\duedate}{August 28, 2020}
\newcommand{\yourname}{Andrew Smith}

% standard AMS math packages
\usepackage{amssymb,amsmath,amsthm}

% problem + solution Environments
\usepackage{environ}
\usepackage{tabto}
\usepackage[svgnames]{xcolor}
\usepackage{framed}
\usepackage{changepage}
\definecolor{shadecolor}{rgb}{0.8,1.00,0.95}
% the solutions are printed if the following is 1
% they are not printed if it is 0
\def\printsolutions{1}
% problem environment
\newcounter{ProblemNum}
\NewEnviron{problem}[1][]{%
\refstepcounter{ProblemNum}
\if\printsolutions1
\begin{shaded}
\else
\bigskip
\fi
\noindent
{\bf Problem%\theProblemNum%
}
\if\printsolutions0
\ifx&#1&\else (#1 points) \fi
\fi
\BODY
\if\printsolutions1
\end{shaded}
\fi
}
% subproblem environment
\newcounter{SubProblemNum}[ProblemNum]
\NewEnviron{subproblem}[1][]{%
\refstepcounter{SubProblemNum}
\if\printsolutions1
\begin{shaded}
\fi
\begin{adjustwidth}{2em}{}
\noindent
%\textbf{(\alph{SubProblemNum})}
\if\printsolutions0
\ifx&#1&\else (#1 points) \fi
\fi
\BODY
\end{adjustwidth}
\if\printsolutions1
\end{shaded}
\fi
}
% solution environment
\NewEnviron{solution}{%
\if\printsolutions1
\begin{oframed}
\noindent
\textbf{Solution:}
\BODY
\end{oframed}
\fi
}

% Header and Footer Information
\usepackage{fancyhdr}
\usepackage{lastpage}
\pagestyle{fancy}
\lhead{\classname}
\chead{\documentname \if\printsolutions1 --- \yourname \fi}
\rhead{\semester}
\lfoot{\if\printsolutions1 Due: \duedate \else Last Revised: \lastrevised\fi}
\cfoot{}
\rfoot{\thepage\ of \pageref{LastPage} }
\renewcommand{\headrulewidth}{0.4pt}
\renewcommand{\footrulewidth}{0.4pt}


% standard commands
\newcommand{\N}{\mathbb{N}}
\newcommand{\Z}{\mathbb{Z}}
\newcommand{\Q}{\mathbb{Q}}
\newcommand{\R}{\mathbb{R}}
\newcommand{\C}{\mathbb{C}}

\begin{document}

\if\printsolutions1
\begin{center}
\textbf{\Large \yourname}
\end{center}
\else
\begin{center}
\textbf{\Large Due: \duedate}
\end{center}
\fi

%%%%%%%%%%%%%%%%%%%%%%%%%%%%%%%%%%%%%%%%%%%%%%%%%%%%%%%%%%%%%%%%%%%%%%%%%%%%

\begin{problem}\textbf{1.4.1}
Using just the axioms, prove that $ad+bc<ac+bd\text{ if } a<b\text{ and } c<d$.
\end{problem}
\begin{solution}\newline
Assume that $a<b$ and $c<d$.\newline 
Note the third ordering axiom (O3) is:
\begin{center}
    For any $a,b\in\R$ if $a<b$ is true, then $a+c<b+c$ is also true for any $c\in\R$. 
\end{center}
Now, by O3, $a<b \implies a-a<b-a\implies 0<b-a.$\newline
Note the fourth ordering axiom (O4) is:
\begin{center}
    For any $a,b\in\R$ if $a<b$ is true, then $a\cdot c<b\cdot c$ is also for any $c\in\R$ for which $c>0$.
\end{center}
Now, by O4, $c<d\implies c\cdot(b-a)<d\cdot(b-a)\implies c\cdot b - c\cdot a < d\cdot b-d
\cdot a$ by the distributive field axiom. (AM1)\newline
Now, by O3, $c\cdot b-c\cdot a<d\cdot b-d\cdot a\implies c\cdot b+d\cdot a<d\cdot b+c\cdot a$.\newline
Now, by the commutativity of multiplication (M1), we have $c\cdot b+d\cdot a<d\cdot b+c\cdot a\implies b\cdot b+a\cdot d<b\cdot d+a\cdot c$.\newline
Now, by the commutativity of addition (A1), we have $a\cdot d+b\cdot c<a\cdot c+b\cdot d$\flushright $\qedsymbol$
\end{solution}
%%%%%%%%%%%%%%%%%%%%%%%%%%%%%%%%%%%%%%%%%%%%%%%%%%%%%%%%%%%%%%%%%%%%%%%%%%%%
\begin{problem}\textbf{1.4.2}
Show for every $n\in\N$ that $n^2\geq n$.
\end{problem}
\begin{solution}\newline
Base Case:\newline
$\underline{n=1}\newline
1^2\geq 1$\newline
Assume $k^2\geq k$ for $k\in\N\newline$
We want to show $(k+1)^2\geq(k+1)$.\newline
$(k+1)^2=k^2+2k+1\newline
\geq k+2k+1$ because $k^2\geq k$ by the induction hypothesis\newline
$=3k+1\newline
\geq k+1$ because $3k>k \iff 2k>0$ which is true because we defined $k\geq 1$.\newline
Thus, we have proven $(k+1)^2\geq(k+1)\newline
\therefore n^2\geq n$ holds for every $n\in\N$\flushright$\qedsymbol$.
\end{solution}
%%%%%%%%%%%%%%%%%%%%%%%%%%%%%%%%%%%%%%%%%%%%%%%%%%%%%%%%%%%%%%%%%%%%%%%%%%%%
\begin{problem}\textbf{1.4.3}
Using just the axioms, prove the \textit{arithmetic-geometric mean inequality}:
\begin{center}
    $\sqrt{ab}\leq\displaystyle\frac{a+b}{2}$
\end{center}
for any $a,b\in\R$ with $a>0$ and $b>0$. (Assume for the moment, the existence of square roots.)
\end{problem}
\begin{solution}\newline
Note for $a,b\geq0, a<b\iff a^2<b^2$\newline\newline
Since $a>0$ and $b>0$, it is clear that $\sqrt{ab}>0$ and $\displaystyle\frac{a+b}{2}>0$\newline
Now this implies, by the first note, that $ab\leq\displaystyle\frac{(a+b)^2}{4}\newline
\implies \displaystyle\frac{(a+b)^2}{4}-ab\geq0\newline
\implies \displaystyle(a+b)^2-4ab\geq0\newline
\implies a^2+b^2-2ab\geq0\newline
\implies(a-b)^2\geq0$ which we know is true, because $a-b\in\R$ and we know that for $r\in\R, r^2>0$\flushright$\qedsymbol$
\end{solution}
%%%%%%%%%%%%%%%%%%%%%%%%%%%%%%%%%%%%%%%%%%%%%%%%%%%%%%%%%%%%%%%%%%%%%%%%%%%%

\end{document}