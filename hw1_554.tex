\documentclass[12pt]{article}
\usepackage[%
	letterpaper,%
	includeheadfoot,%
	margin=0.5in,%
	headheight=0.21in,%
	footskip=0.51in%
]{geometry}

% Heading information
\newcommand{\documentname}{Homework 1}
\newcommand{\classname}{Math 554 --- H01}
\newcommand{\semester}{Fall 2020}
\newcommand{\lastrevised}{\today}
\newcommand{\duedate}{August 24, 2020}
\newcommand{\yourname}{Andrew Smith}

% standard AMS math packages
\usepackage{amssymb,amsmath,amsthm}

% problem + solution Environments
\usepackage{environ}
\usepackage{tabto}
\usepackage[svgnames]{xcolor}
\usepackage{framed}
\usepackage{changepage}
\definecolor{shadecolor}{rgb}{0.8,1.00,0.95}
% the solutions are printed if the following is 1
% they are not printed if it is 0
\def\printsolutions{1}
% problem environment
\newcounter{ProblemNum}
\NewEnviron{problem}[1][]{%
\refstepcounter{ProblemNum}
\if\printsolutions1
\begin{shaded}
\else
\bigskip
\fi
\noindent
{\bf Problem%\theProblemNum%
}
\if\printsolutions0
\ifx&#1&\else (#1 points) \fi
\fi
\BODY
\if\printsolutions1
\end{shaded}
\fi
}
% subproblem environment
\newcounter{SubProblemNum}[ProblemNum]
\NewEnviron{subproblem}[1][]{%
\refstepcounter{SubProblemNum}
\if\printsolutions1
\begin{shaded}
\fi
\begin{adjustwidth}{2em}{}
\noindent
%\textbf{(\alph{SubProblemNum})}
\if\printsolutions0
\ifx&#1&\else (#1 points) \fi
\fi
\BODY
\end{adjustwidth}
\if\printsolutions1
\end{shaded}
\fi
}
% solution environment
\NewEnviron{solution}{%
\if\printsolutions1
\begin{oframed}
\noindent
\textbf{Solution:}
\BODY
\end{oframed}
\fi
}

% Header and Footer Information
\usepackage{fancyhdr}
\usepackage{lastpage}
\pagestyle{fancy}
\lhead{\classname}
\chead{\documentname \if\printsolutions1 --- \yourname \fi}
\rhead{\semester}
\lfoot{\if\printsolutions1 Due: \duedate \else Last Revised: \lastrevised\fi}
\cfoot{}
\rfoot{\thepage\ of \pageref{LastPage} }
\renewcommand{\headrulewidth}{0.4pt}
\renewcommand{\footrulewidth}{0.4pt}


% standard commands
\newcommand{\N}{\mathbb{N}}
\newcommand{\Z}{\mathbb{Z}}
\newcommand{\Q}{\mathbb{Q}}
\newcommand{\R}{\mathbb{R}}
\newcommand{\C}{\mathbb{C}}

\begin{document}

\if\printsolutions1
\begin{center}
\textbf{\Large \yourname}
\end{center}
\else
\begin{center}
\textbf{\Large Due: \duedate}
\end{center}
\fi

%%%%%%%%%%%%%%%%%%%%%%%%%%%%%%%%%%%%%%%%%%%%%%%%%%%%%%%%%%%%%%%%%%%%%%%%%%%%

\begin{problem}
\textbf{A.2.1} This exercise introduces the idea of set equality. The identity $X = Y$ for sets means that they have identical elements. To prove such an assertion assume first that $x \in X$ is any element. Now show that $x \in Y$. Then assume that $y \in Y$ is any element. Now show that $y \in X$.
\end{problem}
\begin{subproblem}\textbf{(a)}
Show that $A\cup B = B$ if and only if $A\subset B$.
\end{subproblem}

\begin{solution}\newline
($\Rightarrow$) Assume that $A\cup B = B$.\newline
Take $a\in A$.\newline
By definition, $a\in A\cup B$.\newline
By hypothesis, $a\in B$.\newline
$\implies A\subset B$.\newline
($\Leftarrow$) Assume that $A\subset B$.\newline
Take $a\in A\cup B$.\newline
By hypothesis, $a\in B$.\newline
$\implies A\cup B\subset B$.\newline
Now, take $b\in B$.\newline
By definition, $b\in A\cup B$.\newline
$\implies B\subset A\cup B$.\newline
$\therefore A\cup B = B$\flushright $\qedsymbol$
\end{solution}
%%%%%%%%%%%%%%%%%%%%%%%%%%%%%%%%%%%%%%%%%%%%%%%%%%%%%%%%%%%%%%%%%%%%%%%%%%%%
\newpage
\begin{subproblem}\textbf{(b)}
Show that $A \cap B = A$ if and only if $A \subset B$.
\end{subproblem}

\begin{solution}\newline
($\Rightarrow$) Assume that $A\cap B = A$.\newline
Take $a\in A$.\newline
By the hypothesis, $a\in A\cap B\implies a\in B \implies A\subset B$.\newline
($\Leftarrow$) Assume that $A\subset B$.\newline
Take $a\in A\cap B$.\newline
By definition, $a\in A$.\newline
$\implies A\cap B\subset A$.\newline
Now, take $b\in A$.\newline
By hypothesis, $b\in B$.\newline
$\implies b\in A\cap B$.\newline
$\implies A\subset A\cap B$.\newline
$\therefore A\cap B = A$\flushright \qedsymbol
\end{solution}
%%%%%%%%%%%%%%%%%%%%%%%%%%%%%%%%%%%%%%%%%%%%%%%%%%%%%%%%%%%%%%%%%%%%%%%%%%%%
\begin{subproblem}\textbf{(c)}
Show that $(A \cup B) \cap C = (A \cap C)\cup (B \cap C)$.
\end{subproblem}

\begin{solution}\newline
Take $x\in (A\cup B)\cap C$.\newline
So $x\in A\cup B\land x\in C$.\newline
Thus $(x\in A \lor x\in B) \land x\in C$.\newline
Thus $(x\in A\land x\in C)\lor (x\in B\land x\in C)$.\newline
Thus $(x\in A\cap C) \lor (x \in B\cap C)$.\newline
Thus $x\in (A\cap C)\cup (B\cap C)$.\newline
Thus $(A\cup B)\cap C\subset (A\cap C)\cup (B\cap C)$.\newline\newline
Now, take $y\in(A\cap C)\cup (B\cap C)$\newline
Thus, $y\in (A\cap C)\lor y\in (B\cap C)$\newline
Thus, $(y\in A \land y\in C) \lor (y\in B\land y\in C)$\newline
Thus, $(y\in A\lor y\in B) \land y\in C$.\newline
Thus, $y\in A\cup B \land y\in C$.\newline
Thus, $y\in (A\cup B)\cap C$.\newline
Thus, $(A\cap C)\cup (B\cap C) \subset (A\cup B) \cap C$.\newline
$\therefore (A\cup B) \cap C = (A\cap C)\cup (B\cap C)$\flushright$\qedsymbol$
\end{solution}
%%%%%%%%%%%%%%%%%%%%%%%%%%%%%%%%%%%%%%%%%%%%%%%%%%%%%%%%%%%%%%%%%%%%%%%%%%%%
\newpage
\begin{subproblem}\textbf{(f)}
Show that $(A \cap B) \setminus C = (A \setminus C) \cap (B \setminus C)$.
\end{subproblem}

\begin{solution}
\newline Take $x\in (A\cap B)\setminus C$.\newline
Thus, $x\in A\cap B\land x\notin C$.\newline
Thus, $(x\in A \land x\in B) \land x\notin C$.\newline
Thus, $(x\in A \land x\notin C)\land(x\in B\land x\notin C)$.\newline
Thus, $x\in A\setminus C\land x\in B\setminus C$.\newline
Thus, $x\in (A\setminus C)\cap (B\setminus C)$.\newline
Thus, $(A\cap B)\setminus C \subset (A\setminus C)\cap (B\setminus C)$.\newline\newline
Now, take $y\in (A\setminus C)\cap (B\setminus C)$.\newline
Thus, $y\in(A\setminus C)\land y\in (B\setminus C)$.\newline
Thus, $(y\in A \land y\notin C)\land (y\in B \land y\notin C)$\newline
Thus, $(y\in A \land y\in B)\land y\notin C$\newline
Thus, $y\in A\cap B \land y\notin C$\newline
Thus, $y\in (A\cap B)\setminus C$\newline
Thus, $(A\setminus C)\cap (B\setminus C)\subset (A\cap B)\setminus C$.\newline
$\therefore(A\cap B)\setminus C = (A\setminus C)\cap (B\setminus C)$\flushright$\qedsymbol$
\end{solution}
%%%%%%%%%%%%%%%%%%%%%%%%%%%%%%%%%%%%%%%%%%%%%%%%%%%%%%%%%%%%%%%%%%%%%%%%%%%%
\begin{subproblem}\textbf{(g)}
Show that $\{x \in \R : x^2 + x < 0\} = (-1,0)$.
\end{subproblem}

\begin{solution}\newline
Let $A = \{ x\in \R : x^2 + x < 0\}$. We want to show that $A=(-1,0)$.\newline
Take $a\in A\implies a^2+a<0\implies a(a+1)<0$.\newline
Now, the zeros of this inequality are $\{-1,0\}$.\newline
Thus, by plugging in a value from each range separated by the zeros, we find the following:\newline 
$(-\infty,-1)$ has a positive range\newline
$(-1,0)$ has a negative range\newline
$(0,\infty)$ has a positive range.\newline
Thus, the domain which satisfies the inequality $a^2+a<0$ is $(-1,0)$.\newline
$\therefore A= (-1,0)$.\flushright$\qedsymbol$
\end{solution}
%%%%%%%%%%%%%%%%%%%%%%%%%%%%%%%%%%%%%%%%%%%%%%%%%%%%%%%%%%%%%%%%%%%%%%%%%%%%
\newpage
\begin{problem}
\textbf{A.2.2} This exercise introduces the notations $\bigcup_{n=1}^N A_i$ and $\bigcap_{n=1}^N A_i$ for the union and intersection of the sets $A_1, A_2, ..., A_N:$
\end{problem}
\begin{subproblem}\textbf{(a)}
Describe the sets\newline
\begin{center}
$\displaystyle\bigcup_{n=1}^N(-1/n,1/n)$ and $\displaystyle\bigcap_{n=1}^N(-1/n,1/n)$.
\end{center}
\end{subproblem}
\begin{solution}
\newline Notice that $\displaystyle\bigcup_{n=1}^N(-1/n,1/n)=(-1,1)\cup (-1/2,1/2)\cup \dots \cup (-1/N,1/N)$.\newline
Notice that $\forall m,n\in \N$ such that $n>m$, $(-1/n,/1/n)\subset(-1/m,1/m)$.\newline
We know that $A\subset B\implies A\cup B = B$.\newline
So, the union collapses by the previous statements.\newline
First, $(-1/N,1/N)\subset (\frac{-1}{N-1},\frac{1}{N-1})$; therefore, $(\frac{-1}{N},\frac{1}{N})\cup(\frac{-1}{N-1},\frac{1}{N-1})=(\frac{-1}{N-1},\frac{1}{N-1})$.\newline
Therefore, iterate this idea and we have that $\displaystyle\bigcup_{n=1}^N(-1/n,1/n)=(-1,1)$.
\newline\newline\newline
Now, notice that $\displaystyle\bigcap_{n=1}^N(-1/n,1/n) = $
$(\frac{-1}{1},\frac{1}{1})\cap(\frac{-1}{2},\frac{1}{2})\cap\dots\cap(\frac{-1}{N},\frac{1}{N})$.\newline
Notice that $\forall m,n\in \N$ such that $n>m$, $(-1/n,/1/n)\subset(-1/m,1/m)$.\newline
We know that $A\subset B\implies A\cap B = A$.\newline
So, the intersection collapses by the previous statements.\newline
First, $(-1/N,1/N)\subset (\frac{-1}{N-1},\frac{1}{N-1})$; therefore, $(\frac{-1}{N},\frac{1}{N})\cap(\frac{-1}{N-1},\frac{1}{N-1})=(\frac{-1}{N},\frac{1}{N})$.\newline
Therefore, iterate this idea and we have that $\displaystyle\bigcap_{n=1}^N(-1/n,1/n) = $
$(\frac{-1}{N},\frac{1}{N})$
\end{solution}
%%%%%%%%%%%%%%%%%%%%%%%%%%%%%%%%%%%%%%%%%%%%%%%%%%%%%%%%%%%%%%%%%%%%%%%%%%%%
\newpage
\begin{subproblem}\textbf{(b)}
Describe the sets\newline
\begin{center}
$\displaystyle\bigcup_{n=1}^N(-n,n)$ and $\displaystyle\bigcap_{n=1}^N(-n,n)$.
\end{center}
\end{subproblem}
\begin{solution}
\newline $\displaystyle\bigcup_{n=1}^N(-n,n) = \textstyle (-1,1)\cup (-2,2) \cup \dots \cup (-N,N)$.\newline
Notice that $\forall m,n\in \N$ such that $n>m, (-n,n)\cup(-m,m) = (-n,n)$.\newline
Iterate this idea, and the previous union collapses:
\newline $\displaystyle\bigcup_{n=1}^N(-n,n)=(-N,N)$.\newline\newline\newline
$\displaystyle\bigcap_{n=1}^N(-n,n) = (-1,1)\cap (-2,2)\cap\dots\cap(-N,N)$.\newline
Notice that $\forall m,n\in \N$ such that $n>m, (-n,n)\cap(-m,m) = (-m,m)$.\newline
Iterate this idea, and the previous intersection collapses: \newline $\displaystyle\bigcap_{n=1}^N(-n,n) = \textstyle (-1,1)$
\end{solution}
%%%%%%%%%%%%%%%%%%%%%%%%%%%%%%%%%%%%%%%%%%%%%%%%%%%%%%%%%%%%%%%%%%%%%%%%%%%%
\begin{subproblem}\textbf{(c)}
Describe the sets\newline
\begin{center}
$\displaystyle\bigcup_{n=1}^N[n,n+1]$ and $\displaystyle\bigcap_{n=1}^N[n,n+1]$.
\end{center}
\end{subproblem}
\begin{solution}
\newline$\displaystyle\bigcup_{n=1}^N[n,n+1] = [1,2]\cup[2,3]\cup\dots\cup[N,N+1] = [1,N+1]$
\newline\newline\newline$\displaystyle\bigcap_{n=1}^N[n,n+1]=[1,2]\cap[2,3]\cap\dots\cap[N,N+1]=\varnothing$

\end{solution}
%%%%%%%%%%%%%%%%%%%%%%%%%%%%%%%%%%%%%%%%%%%%%%%%%%%%%%%%%%%%%%%%%%%%%%%%%%%%
\newpage
\begin{problem}
\textbf{A.2.3} This exercise introduces the notations $\bigcup_{n=1}^\infty A_i$ and $\bigcap_{n=1}^\infty A_i$ for the union and intersection of the sets $A_1, A_2, ...$
\end{problem}
\begin{subproblem}\textbf{(a)}
Describe the sets\newline
\begin{center}
$\displaystyle\bigcup_{n=1}^\infty(-1/n,1/n)$ and $\displaystyle\bigcap_{n=1}^\infty(-1/n,1/n)$.
\end{center}
\end{subproblem}
\begin{solution}
\newline$\displaystyle\bigcup_{n=1}^\infty(-1/n,1/n) = \textstyle(-1,1)\cup(-1/2,1/2)\cup\dots\cup(\frac{-1}{\infty},\frac{1}{\infty}) = (-1,1)$
\newline\newline\newline $\displaystyle\bigcap_{n=1}^\infty(-1/n,1/n) = (-1,1)\cap(-1/2,1/2)\cap\dots\cap(-1/\infty,1/\infty) = \varnothing$
\end{solution}
%%%%%%%%%%%%%%%%%%%%%%%%%%%%%%%%%%%%%%%%%%%%%%%%%%%%%%%%%%%%%%%%%%%%%%%%%%%%
\begin{subproblem}\textbf{(b)}
Describe the sets\newline
\begin{center}
$\displaystyle\bigcup_{n=1}^\infty(-n,n)$ and $\displaystyle\bigcap_{n=1}^\infty(-n,n)$.
\end{center}
\end{subproblem}
\begin{solution}
\newline$\displaystyle\bigcup_{n=1}^\infty(-n,n)=(-1,1)\cup(-2,2)\cup\dots\cup(-\infty,\infty) = (-\infty,\infty)$.
\newline\newline\newline $\displaystyle\bigcap_{n=1}^\infty(-n,n)=(-1,1)\cap(-2,2)\cap\dots\cap(-\infty,\infty)=(-1,1)$
\end{solution}
%%%%%%%%%%%%%%%%%%%%%%%%%%%%%%%%%%%%%%%%%%%%%%%%%%%%%%%%%%%%%%%%%%%%%%%%%%%%
\newpage
\begin{subproblem}\textbf{(c)}
Describe the sets\newline
\begin{center}
$\displaystyle\bigcup_{n=1}^\infty[n,n+1]$ and $\displaystyle\bigcap_{n=1}^\infty[n,n+1]$.
\end{center}
\end{subproblem}
\begin{solution}\newline
$\displaystyle\bigcup_{n=1}^\infty[n,n+1]=[1,2]\cup[2,3]\cup\dots\cup[\infty,\infty+1]=[1,\infty)$
\newline\newline\newline$\displaystyle\bigcap_{n=1}^\infty[n,n+1]=[1,2]\cap[2,3]\cap\dots\cap[\infty,\infty+1]=\varnothing
$\newline\newline NOTE: I do not think the convention $[\infty,\infty+1]$ is correct; however, it is neither here nor there for these examples.
\end{solution}
%%%%%%%%%%%%%%%%%%%%%%%%%%%%%%%%%%%%%%%%%%%%%%%%%%%%%%%%%%%%%%%%%%%%%%%%%%%%
\end{document}