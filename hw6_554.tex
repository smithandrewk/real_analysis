\documentclass[12pt]{article}
\usepackage[%
	letterpaper,%
	includeheadfoot,%
	margin=0.5in,%
	headheight=0.21in,%
	footskip=0.51in%
]{geometry}

% Heading information
\newcommand{\documentname}{Homework 06}
\newcommand{\classname}{Math 554 --- H01}
\newcommand{\semester}{Fall 2020}
\newcommand{\lastrevised}{\today}
\newcommand{\duedate}{September 4, 2020}
\newcommand{\yourname}{Andrew Smith}

% standard AMS math packages
\usepackage{amssymb,amsmath,amsthm}

% problem + solution Environments
\usepackage{environ}
\usepackage{tabto}
\usepackage[svgnames]{xcolor}
\usepackage{framed}
\usepackage{changepage}
\definecolor{shadecolor}{rgb}{0.8,1.00,0.95}
% the solutions are printed if the following is 1
% they are not printed if it is 0
\def\printsolutions{1}
% problem environment
\newcounter{ProblemNum}
\NewEnviron{problem}[1][]{%
\refstepcounter{ProblemNum}
\if\printsolutions1
\begin{shaded}
\else
\bigskip
\fi
\noindent
{\bf Problem%\theProblemNum%
}
\if\printsolutions0
\ifx&#1&\else (#1 points) \fi
\fi
\BODY
\if\printsolutions1
\end{shaded}
\fi
}
% subproblem environment
\newcounter{SubProblemNum}[ProblemNum]
\NewEnviron{subproblem}[1][]{%
\refstepcounter{SubProblemNum}
\if\printsolutions1
\begin{shaded}
\fi
\begin{adjustwidth}{2em}{}
\noindent
%\textbf{(\alph{SubProblemNum})}
\if\printsolutions0
\ifx&#1&\else (#1 points) \fi
\fi
\BODY
\end{adjustwidth}
\if\printsolutions1
\end{shaded}
\fi
}
% solution environment
\NewEnviron{solution}{%
\if\printsolutions1
\begin{oframed}
\noindent
\textbf{Solution:}
\BODY
\end{oframed}
\fi
}

% Header and Footer Information
\usepackage{fancyhdr}
\usepackage{lastpage}
\pagestyle{fancy}
\lhead{\classname}
\chead{\documentname \if\printsolutions1 --- \yourname \fi}
\rhead{\semester}
\lfoot{\if\printsolutions1 Due: \duedate \else Last Revised: \lastrevised\fi}
\cfoot{}
\rfoot{\thepage\ of \pageref{LastPage} }
\renewcommand{\headrulewidth}{0.4pt}
\renewcommand{\footrulewidth}{0.4pt}


% standard commands
\newcommand{\N}{\mathbb{N}}
\newcommand{\Z}{\mathbb{Z}}
\newcommand{\Q}{\mathbb{Q}}
\newcommand{\R}{\mathbb{R}}
\newcommand{\C}{\mathbb{C}}

\begin{document}

\if\printsolutions1
\begin{center}
\textbf{\Large \yourname}
\end{center}
\else
\begin{center}
\textbf{\Large Due: \duedate}
\end{center}
\fi

%%%%%%%%%%%%%%%%%%%%%%%%%%%%%%%%%%%%%%%%%%%%%%%%%%%%%%%%%%%%%%%%%%%%%%%%%%%%
\begin{problem}\textbf{1.9.1}
	Show that the definition of ``dense'' could be given as
	\begin{center}
		A set $E$ of real numbers is said to be \textit{dense} if every interval $(a,b)$ contains infinitely many points of $E$.
	\end{center}
\end{problem}
\begin{solution}\newline
Assume $E$ is dense. Take an interval $(a,b)$.\newline
By definition, we know $\exists c_1\in E$ such that $c_1\in(a,b)$.\newline
$\implies(a,c_1)\subset(a,b)$.\newline
Now, we know that $\exists c_2\in E$ such that $c_2\in(a,c_1)$.\newline
$\implies(a,c_2)\subset(a,c_1)$.\newline
	Assume $\exists c_{n-1}\in E$ such that $c_{n-1}\in (a,c_{n-2})\newline
	\implies (a,c_{n-1})\subset (a,c_{n-2})\newline$
	Therefore, since $E$ is dense and $(a,c_{n-1})\in E$, $\exists c_n\in E$ such that $c_n\in(a,c_{n-1})\newline
	\implies\{c_n:n\in\N\}\subset(a,b)\cap E\newline
	\therefore(a,b)$ contains inifinitely many points of $E$\flushright$\qedsymbol$


\end{solution}
%%%%%%%%%%%%%%%%%%%%%%%%%%%%%%%%%%%%%%%%%%%%%%%%%%%%%%%%%%%%%%%%%%%%%%%%%%%%
\begin{problem}\textbf{1.9.3}
If a set $E$ is dense, what can you conclude about a set $A\supset E$?
\end{problem}
\begin{solution}\newline
$E$ is dense.\newline
Therefore, for all $a,b\in\R,a<b,(a,b)\cap E\neq\varnothing$.\newline
	Since $E\subset A$, we have for all $a,b\in\R,a<b,(a,b)\cap A\neq\varnothing$.\newline
	Therefore, $A$ is dense.\flushright$\qedsymbol$
\end{solution}
%%%%%%%%%%%%%%%%%%%%%%%%%%%%%%%%%%%%%%%%%%%%%%%%%%%%%%%%%%%%%%%%%%%%%%%%%%%%
\begin{problem}\textbf{1.9.5}
If two sets $E_1$ and $E_2$ are dense, what can you conclude about the set $E_1\cap E_2$?
\end{problem}
\begin{solution}\newline
	We cannot conclude much in general about $E_1\cap E_2$.\newline
Take, for exmaple $E_1=\Q$ and $E_2=\R\setminus\Q$.\newline
Now, we have proven before that both of these sets are dense; however $E_1\cap E_2=\varnothing$ is not dense.\newline
Take, for another example, $E_1=E_2=\Q\implies E_1\cap E_2=\Q\implies E_1\cap E_2$ is dense.\newline
Therefore, $E_1,E_2$ dense does not necessarily imply $E_1\cap E_2$ dense.
\end{solution}
\end{document}
