\documentclass[12pt]{article}
\usepackage[%
	letterpaper,%
	includeheadfoot,%
	margin=0.5in,%
	headheight=0.21in,%
	footskip=0.51in%
]{geometry}

% Heading information
\newcommand{\documentname}{Homework 8}
\newcommand{\classname}{Math 554 --- H01}
\newcommand{\semester}{Fall 2020}
\newcommand{\lastrevised}{\today}
\newcommand{\duedate}{September 11, 2020}
\newcommand{\yourname}{Andrew Smith}

% standard AMS math packages
\usepackage{amssymb,amsmath,amsthm}

% problem + solution Environments
\usepackage{environ}
\usepackage{tabto}
\usepackage[svgnames]{xcolor}
\usepackage{framed}
\usepackage{changepage}
\definecolor{shadecolor}{rgb}{0.8,1.00,0.95}
% the solutions are printed if the following is 1
% they are not printed if it is 0
\def\printsolutions{1}
% problem environment
\newcounter{ProblemNum}
\NewEnviron{problem}[1][]{%
\refstepcounter{ProblemNum}
\if\printsolutions1
\begin{shaded}
\else
\bigskip
\fi
\noindent
{\bf Problem%\theProblemNum%
}
\if\printsolutions0
\ifx&#1&\else (#1 points) \fi
\fi
\BODY
\if\printsolutions1
\end{shaded}
\fi
}
% subproblem environment
\newcounter{SubProblemNum}[ProblemNum]
\NewEnviron{subproblem}[1][]{%
\refstepcounter{SubProblemNum}
\if\printsolutions1
\begin{shaded}
\fi
\begin{adjustwidth}{2em}{}
\noindent
%\textbf{(\alph{SubProblemNum})}
\if\printsolutions0
\ifx&#1&\else (#1 points) \fi
\fi
\BODY
\end{adjustwidth}
\if\printsolutions1
\end{shaded}
\fi
}
% solution environment
\NewEnviron{solution}{%
\if\printsolutions1
\begin{oframed}
\noindent
\textbf{Solution:}
\BODY
\end{oframed}
\fi
}

% Header and Footer Information
\usepackage{fancyhdr}
\usepackage{lastpage}
\pagestyle{fancy}
\lhead{\classname}
\chead{\documentname \if\printsolutions1 --- \yourname \fi}
\rhead{\semester}
\lfoot{\if\printsolutions1 Due: \duedate \else Last Revised: \lastrevised\fi}
\cfoot{}
\rfoot{\thepage\ of \pageref{LastPage} }
\renewcommand{\headrulewidth}{0.4pt}
\renewcommand{\footrulewidth}{0.4pt}


% standard commands
\newcommand{\N}{\mathbb{N}}
\newcommand{\Z}{\mathbb{Z}}
\newcommand{\Q}{\mathbb{Q}}
\newcommand{\R}{\mathbb{R}}
\newcommand{\C}{\mathbb{C}}

\begin{document}

\if\printsolutions1
\begin{center}
\textbf{\Large \yourname}
\end{center}
\else
\begin{center}
\textbf{\Large Due: \duedate}
\end{center}
\fi

%%%%%%%%%%%%%%%%%%%%%%%%%%%%%%%%%%%%%%%%%%%%%%%%%%%%%%%%%%%%%%%%%%%%%%%%%%%%

\begin{problem}\textbf{2.2.8}
Consider the sequence defined recursively by
\begin{center}
    $x_1=\sqrt{2},x_n=\sqrt{2+x_{n-1}}$.
\end{center}
Show, by induction, that $x_n<2$ for all $n$.
\end{problem}
\begin{solution}\newline
\underline{Base Case.}\newline
Let $n=1$. $x_1=\sqrt{2}<2$.\newline
Let $n=2$. $x_2=\sqrt{2+x_1}=\sqrt{2+\sqrt{2}}\iff (x_2)^2=2+\sqrt{2}<2+2$ by case $n=1$.\newline
$(x_2)^2<4$ implies that $x_2<2$.\newline
\underline{Inductive Hypothesis.}\newline
Assume that $x_n<2$ for $n$.\newline
We want to show that $x_{n+1}<2$.\newline
$x_{n+1}=\sqrt{2+x_n}\iff (x_{n+1})^2=2+x_n<2+2$ by the inductive hypothesis.\newline
$(x_{n+1})^2<4$ implies that $x_{n+1}<2$.\flushright$\qedsymbol$
\end{solution}
%%%%%%%%%%%%%%%%%%%%%%%%%%%%%%%%%%%%%%%%%%%%%%%%%%%%%%%%%%%%%%%%%%%%%%%%%%%%

\begin{problem}\textbf{2.2.9}
Consider the sequence defined recursively by
\begin{center}
    $x_1=\sqrt{2},x_n=\sqrt{2+x_{n-1}}$.
\end{center}
Show, by induction, that $x_n<x_{n+1}$ for all $n$.
\end{problem}
\begin{solution}\newline
\underline{Base Case.}\newline
Let $n=1$.\newline 
We want to prove the inequality $x_1<x_2$.\newline
It will suffice to prove $(x_1)^2<(x_2)^2$ since $x_1,x_2>0$.\newline
The previous inequality evaluates to $2<2+\sqrt{2}$.\newline
This implies $0<\sqrt{2}$, which is true.\newline
Therefore, $x_1<x_2$.\newline
\underline{Inductive Hypothesis.}\newline
Assume that $x_n<x_{n+1}$ for $n$.\newline
We want to show $x_{n+1}<x_{n+2}$.\newline
It will suffice to show $(x_{n+1})^2<(x_{n+2})^2$ since $x_{n+1},x_{n+2}>0$.\newline
Substituting in, we want $2+x_n<2+x_{n+1}\iff x_n<x_{n+1}$, which is true by the inductive hypothesis.\newline
Therefore, $x_{n+1}<x_{n+2}.$\flushright$\qedsymbol$

\end{solution}
%%%%%%%%%%%%%%%%%%%%%%%%%%%%%%%%%%%%%%%%%%%%%%%%%%%%%%%%%%%%%%%%%%%%%%%%%%%%

\begin{problem}\textbf{2.4.11}
The sequence $s_n=(-1)^n$ does not converge. For what values of $\varepsilon>0$ is it nonetheless true that there is an integer $N$ so that $|s_n-1|<\varepsilon$ whenever $n\geq N$? For what values of $\varepsilon>0$ is it nonetheless true that there is an integer $N$ so that $|s_n-0|<\varepsilon$ whenever $n\geq N$?
\end{problem}
\begin{solution}\newline
$s_n=(-1)^n={-1,1,-1,...}$ oscillates between $-1$ and $1$.\newline
So, $s_n=-1\lor s_n=1\forall n\in N$.\newline
So, $|s_n-1|=2\lor|s_n-1|=0\forall n\in N$.\newline
Therefore, for $\varepsilon>2$, it is nonetheless true that there is an integer $N$ so that $|s_n-1|<\varepsilon$ whenever $n\geq N$.\newline\newline
Similarly, $|s_n-0|=1\lor |s_n-0|=1\forall n\in \N$.\newline
Therefore, for $\varepsilon>1$, it is nonetheless true that there is an integer $N$ so that $|s_n-0|<\varepsilon$ whenever $n\geq N$.
\end{solution}
%%%%%%%%%%%%%%%%%%%%%%%%%%%%%%%%%%%%%%%%%%%%%%%%%%%%%%%%%%%%%%%%%%%%%%%%%%%%

\begin{problem}\textbf{2.4.12}
Let $\{s_n\}$ be a sequence that assumes only integer values. Under what conditions can such a sequence converge?
\end{problem}
\begin{solution}\newline
A sequence $\{s_n\}$ converges if for all $\varepsilon>0$, there exists $N\in\N$ such that $|s_n-L|<\varepsilon$ for all $n\geq N$.\newline
Let $\varepsilon=.5$.\newline
Now, we know that there exists an $N\in\N$ such that $|s_n-L|<.5$ for all $n\geq N$.\newline
Take $n,m\geq N$.\newline
$|s_n-s_m|=|s_n-L+L-s_m|=|(s_n-L)+(L-s_m)|\leq |s_n-L|+|L-s_m|=|s_n-L|-|s_m-L|<1$.\newline
Since $s_n$ and $s_m$ are integers, $|s_n-s_m|=0$.\newline
Therefore we see that $s_n=s_m$ for all $n\geq N$.\newline
So $\{s_n\}$ converges to an integer $L$ for large enough $n$.\flushright$\qedsymbol$
\end{solution}
%%%%%%%%%%%%%%%%%%%%%%%%%%%%%%%%%%%%%%%%%%%%%%%%%%%%%%%%%%%%%%%%%%%%%%%%%%%%
\newpage
\begin{problem}\textbf{2.5.8}
Suppose that a sequence $\{s_n\}$ of positive numbers satisfies the condition $s_{n+1}>\alpha s_n$ for all $n$ where $\alpha>1$. Show that $s_n\rightarrow\infty$.
\end{problem}
\begin{solution}\newline
Assume that $\{s_n\},s_n>0$ satisfies that condition $s_{n+1}>\alpha s_n$ for all $n$ where $\alpha>1$.\newline
We want to show that $s_n\rightarrow\infty$\newline
$s_n\rightarrow\infty\iff \displaystyle{\lim_{n \to \infty}s_n=\infty}\iff$ for $M\in \R,\exists N\in\N$ such that $s_n>M$ for all $n\geq N$.\newline
Let $M=\alpha s_n\in\R,N=n+1$.\newline
We need to show that $s_{n+1}>M$ for all $n\geq N$.\newline
\underline{Base Case.}\newline
$s_{n+1}>M$ by assumption.\newline
\underline{Inductive Hypothesis.}\newline
We want to show that $s_{n+2}>M$.\newline
We know that $s_{n+2}>\alpha s_{n+1}>\alpha\alpha s_n\iff s_{n+2}>s_{n+1}>\alpha s_n=M$.\newline
Therefore, $s_n\rightarrow\infty$.\flushright$\qedsymbol$


\end{solution}


%%%%%%%%%%%%%%%%%%%%%%%%%%%%%%%%%%%%%%%%%%%%%%%%%%%%%%%%%%%%%%%%%%%%%%%%%%%%

\end{document}