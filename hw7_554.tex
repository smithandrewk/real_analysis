\documentclass[12pt]{article}
\usepackage[%
	letterpaper,%
	includeheadfoot,%
	margin=0.5in,%
	headheight=0.21in,%
	footskip=0.51in%
]{geometry}

% Heading information
\newcommand{\documentname}{Homework 7 (How to write proofs)}
\newcommand{\classname}{Math 554 --- H01}
\newcommand{\semester}{Fall 2020}
\newcommand{\lastrevised}{\today}
\newcommand{\duedate}{September 9, 2020}
\newcommand{\yourname}{Andrew Smith}

% standard AMS math packages
\usepackage{amssymb,amsmath,amsthm}

% problem + solution Environments
\usepackage{environ}
\usepackage{tabto}
\usepackage[svgnames]{xcolor}
\usepackage{framed}
\usepackage{changepage}
\definecolor{shadecolor}{rgb}{0.8,1.00,0.95}
% the solutions are printed if the following is 1
% they are not printed if it is 0
\def\printsolutions{1}
% problem environment
\newcounter{ProblemNum}
\NewEnviron{problem}[1][]{%
\refstepcounter{ProblemNum}
\if\printsolutions1
\begin{shaded}
\else
\bigskip
\fi
\noindent
{\bf Problem%\theProblemNum%
}
\if\printsolutions0
\ifx&#1&\else (#1 points) \fi
\fi
\BODY
\if\printsolutions1
\end{shaded}
\fi
}
% subproblem environment
\newcounter{SubProblemNum}[ProblemNum]
\NewEnviron{subproblem}[1][]{%
\refstepcounter{SubProblemNum}
\if\printsolutions1
\begin{shaded}
\fi
\begin{adjustwidth}{2em}{}
\noindent
%\textbf{(\alph{SubProblemNum})}
\if\printsolutions0
\ifx&#1&\else (#1 points) \fi
\fi
\BODY
\end{adjustwidth}
\if\printsolutions1
\end{shaded}
\fi
}
% solution environment
\NewEnviron{solution}{%
\if\printsolutions1
\begin{oframed}
\noindent
\textbf{Solution:}
\BODY
\end{oframed}
\fi
}

% Header and Footer Information
\usepackage{fancyhdr}
\usepackage{lastpage}
\pagestyle{fancy}
\lhead{\classname}
\chead{\documentname \if\printsolutions1 --- \yourname \fi}
\rhead{\semester}
\lfoot{\if\printsolutions1 Due: \duedate \else Last Revised: \lastrevised\fi}
\cfoot{}
\rfoot{\thepage\ of \pageref{LastPage} }
\renewcommand{\headrulewidth}{0.4pt}
\renewcommand{\footrulewidth}{0.4pt}


% standard commands
\newcommand{\N}{\mathbb{N}}
\newcommand{\Z}{\mathbb{Z}}
\newcommand{\Q}{\mathbb{Q}}
\newcommand{\R}{\mathbb{R}}
\newcommand{\C}{\mathbb{C}}

\begin{document}

\if\printsolutions1
\begin{center}
\textbf{\Large \yourname}
\end{center}
\else
\begin{center}
\textbf{\Large Due: \duedate}
\end{center}
\fi

%%%%%%%%%%%%%%%%%%%%%%%%%%%%%%%%%%%%%%%%%%%%%%%%%%%%%%%%%%%%%%%%%%%%%%%%%%%%

\begin{problem}\textbf{1}
If you were writing a proof of "All prime numbers greater than 2 are odd", which of the following would be appropriate ways to begin the proof? (There may be more than one correct answer).\newline
\indent a. Let \textit{n} be an odd prime number.\newline
\indent b. Assume that all odd prime numbers are greater than 2.\newline
\indent c. Let \textit{n} be a prime number greater than 2.\newline
\indent d. Assume that \textit{n} and \textit{k} are integers with $n>k>2$.\newline
\indent e. The numbers 3,5,7, and 11 are prime numbers greater than 2.\newline
\indent f. Let \textit{n} be a number greater than 2 which is not prime.
\end{problem}
\begin{solution}\newline
C
\end{solution}
\begin{problem}
Write an appropriate first sentence that would begin proofs of each of the following statements:
\end{problem}
\begin{subproblem}\textbf{2.} If \textit{m} and \textit{n} are relatively prime integers, then there exist integers \textit{x} and \textit{y} such that $mx+ny=1$.
\end{subproblem}
\begin{solution}
Assume \textit{m} and \textit{n} are relatively prime integers.
\end{solution}
\begin{subproblem}\textbf{3.}
If \textit{a} and \textit{b} are real numbers with $a\leq b$, and \textit{f} is a function continuous on the closed interval $[a,b]$, then there is a real number \textit{M} such that $|f(x)|\leq M$ for all $x\in [a,b]$.
\end{subproblem}
\begin{solution}
Assume \textit{a} and \textit{b} are real numbers with $a\leq b$, and \textit{f} is a function continuous on the closed interval $[a,b]$.
\end{solution}
%%%%%%%%%%%%%%%%%%%%%%%%%%%%%%%%%%%%%%%%%%%%%%%%%%%%%%%%%%%%%%%%%%%%%%%%%%%%

\end{document}