\documentclass[12pt]{article}
\usepackage[%
	letterpaper,%
	includeheadfoot,%
	margin=0.5in,%
	headheight=0.21in,%
	footskip=0.51in%
]{geometry}

% Heading information
\newcommand{\documentname}{Homework 2}
\newcommand{\classname}{Math 554 --- H01}
\newcommand{\semester}{Fall 2020}
\newcommand{\lastrevised}{\today}
\newcommand{\duedate}{August 26, 2020}
\newcommand{\yourname}{Andrew Smith}

% standard AMS math packages
\usepackage{amssymb,amsmath,amsthm}

% problem + solution Environments
\usepackage{environ}
\usepackage{tabto}
\usepackage[svgnames]{xcolor}
\usepackage{framed}
\usepackage{mathtools}
\usepackage{changepage}
\definecolor{shadecolor}{rgb}{0.8,1.00,0.95}
\DeclarePairedDelimiter{\ceil}{\lceil}{\rceil}
\DeclarePairedDelimiter{\floor}{\lfloor}{\rfloor}
% the solutions are printed if the following is 1
% they are not printed if it is 0
\def\printsolutions{1}
% problem environment
\newcounter{ProblemNum}
\NewEnviron{problem}[1][]{%
\refstepcounter{ProblemNum}
\if\printsolutions1
\begin{shaded}
\else
\bigskip
\fi
\noindent
{\bf Problem%\theProblemNum%
}
\if\printsolutions0
\ifx&#1&\else (#1 points) \fi
\fi
\BODY
\if\printsolutions1
\end{shaded}
\fi
}
% subproblem environment
\newcounter{SubProblemNum}[ProblemNum]
\NewEnviron{subproblem}[1][]{%
\refstepcounter{SubProblemNum}
\if\printsolutions1
\begin{shaded}
\fi
\begin{adjustwidth}{2em}{}
\noindent
%\textbf{(\alph{SubProblemNum})}
\if\printsolutions0
\ifx&#1&\else (#1 points) \fi
\fi
\BODY
\end{adjustwidth}
\if\printsolutions1
\end{shaded}
\fi
}
% solution environment
\NewEnviron{solution}{%
\if\printsolutions1
\begin{oframed}
\noindent
\textbf{Solution:}
\BODY
\end{oframed}
\fi
}

% Header and Footer Information
\usepackage{fancyhdr}
\usepackage{lastpage}
\pagestyle{fancy}
\lhead{\classname}
\chead{\documentname \if\printsolutions1 --- \yourname \fi}
\rhead{\semester}
\lfoot{\if\printsolutions1 Due: \duedate \else Last Revised: \lastrevised\fi}
\cfoot{}
\rfoot{\thepage\ of \pageref{LastPage} }
\renewcommand{\headrulewidth}{0.4pt}
\renewcommand{\footrulewidth}{0.4pt}


% standard commands
\newcommand{\N}{\mathbb{N}}
\newcommand{\Z}{\mathbb{Z}}
\newcommand{\Q}{\mathbb{Q}}
\newcommand{\R}{\mathbb{R}}
\newcommand{\C}{\mathbb{C}}

\begin{document}

\if\printsolutions1
\begin{center}
\textbf{\Large \yourname}
\end{center}
\else
\begin{center}
\textbf{\Large Due: \duedate}
\end{center}
\fi


%%%%%%%%%%%%%%%%%%%%%%%%%%%%%%%%%%%%%%%%%%%%%%%%%%%%%%%%%%%%%%%%%%%%%%%%%%%%
%%%%%%%%%%%%%%%%%%%%%%%%%%%%%%%%%%%%%%%%%%%%%%%%%%%%%%%%%%%%%%%%%%%%%%%%%%%%
\begin{problem}\textbf{A.2.5} This exercise promotes the use of the term \textit{mapping} in the study of functions.\newline
If $f:X\to Y$ and $E\subset X$, then
\begin{center}
    $f(E)=\{y:f(x)=y\text{ for some }x\in E\}\subset Y$
\end{center}
is called the \textit{image} of \textit{E} under \textit{f} and we say \textit{f maps E} to the set \textit{f(E)}.
\end{problem}
%%%%%%%%%%%%%%%%%%%%%%%%%%%%%%%%%%%%%%%%%%%%%%%%%%%%%%%%%%%%%%%%%%%%%%%%%%%%
\begin{subproblem}\textbf{(a)}
Let $f:\R\to\R$. Give an example of sets \textit{A,B} so that
\begin{center}
    $f(A\cap B)\neq f(A)\cap f(B)$.
\end{center}
\end{subproblem}

\begin{solution}\newline
Take $x,y\in\R\text{ such that }x\neq y,f(x)=f(y)$.\newline
Let $A=\{x\}, B=\{y\}$.\newline
$f(A\cap B)=\varnothing$\newline
$f(A)\cap f(B)\neq\varnothing$\newline
$\therefore f(A\cap B)\neq f(A)\cap f(B)$.\flushright NOTE: An example of \textit{f,x,y} is $f:x\mapsto x^2$ where $x=-2,y=2$
\end{solution}
%%%%%%%%%%%%%%%%%%%%%%%%%%%%%%%%%%%%%%%%%%%%%%%%%%%%%%%%%%%%%%%%%%%%%%%%%%%%
\begin{subproblem}\textbf{(b)}
Would $f(A\cup B) = f(A)\cup f(B)\text{ be true in general?}$
\end{subproblem}

\begin{solution}\newline
($\subseteq$) Take $f(x)\in f(A\cup B)\implies\exists x\in A\cup B \text{ such that } f(x)=y$.\newline
$\implies x\in A \lor x \in B$\newline
$\implies f(x)\in f(A)\lor f(x) \in f(B)$\newline
$\implies f(x)\in f(A)\cup f(B)\newline
\implies f(A\cup B)\subset f(A)\cup f(B)$\newline
($\supseteq$) Take $f(x)\in f(A)\cup f(B)\newline
\implies f(x)\in f(A) \lor f(x) \in f(B)\newline
\implies x\in A\lor x\in B\newline
\implies x\in A\cup B\newline
\implies f(x)\in f(A\cup B)\newline
\implies f(A)\cup f(B)\subset f(A\cup B)\newline
\therefore f(A\cup B)=f(A)\cup f(B)$
\end{solution}
%%%%%%%%%%%%%%%%%%%%%%%%%%%%%%%%%%%%%%%%%%%%%%%%%%%%%%%%%%%%%%%%%%%%%%%%%%%%
\begin{subproblem}\textbf{(c)}
Find a function $f:\R\to\R\text{ so that } f([0,1])=\{1,2\}$.
\end{subproblem}

\begin{solution}\newline
$f(x)=\floor{x+1}$
\end{solution}
%%%%%%%%%%%%%%%%%%%%%%%%%%%%%%%%%%%%%%%%%%%%%%%%%%%%%%%%%%%%%%%%%%%%%%%%%%%%
%%%%%%%%%%%%%%%%%%%%%%%%%%%%%%%%%%%%%%%%%%%%%%%%%%%%%%%%%%%%%%%%%%%%%%%%%%%%
\begin{problem}\textbf{A.2.6}
This exercise concerns the notion of one-to-one function (i.e., injective function):
\end{problem}
%%%%%%%%%%%%%%%%%%%%%%%%%%%%%%%%%%%%%%%%%%%%%%%%%%%%%%%%%%%%%%%%%%%%%%%%%%%%
\begin{subproblem}\textbf{(a)}
Show that $f:\R\to\R$ is one-to-one if and only if
\begin{center}
    $f(A\cap B) = f(A)\cap f(B)$
\end{center}
\tab \ for all sets \textit{A,B}.
\end{subproblem}

\begin{solution}\newline
($\Rightarrow$) Assume \textit{f} is injective.\newline
($\subseteq$) Take $y\in f(A\cap B)\implies \exists x\in A\cap B\text{ such that } f(x)=y\newline
\implies x\in A\land x\in B\newline
\implies y\in f(A) \land y\in f(B)\newline
\implies y\in f(A)\cap f(B)\newline
\implies f(A\cap B)\subset f(A)\cap f(B)$\newline
($\supseteq$) Take $f(x)\in f(A) \cap f(B)\newline
\implies f(x)\in f(A)\land f(x)\in f(B)\newline
\implies x\in A\land x\in B\newline
\implies x\in A\cap B\newline
\implies f(x) \in f(A\cap B)\newline
\implies f(A)\cap f(B) \subset f(A\cap B)\newline
\therefore f(A\cap B) = f(A)\cap f(B)$\newline
($\Leftarrow$) Assume $f(A\cap B) = f(A)\cap f(B)\newline$
Take $A,B\in\R$. Assume $f(A)=f(B)$.\newline
$\implies f(A)\cap f(B) \neq \varnothing\newline
\implies f(A\cap B)\neq\varnothing\newline
\implies A=B\newline
\therefore f$ is injective.\flushright$\qedsymbol$
\end{solution}
%%%%%%%%%%%%%%%%%%%%%%%%%%%%%%%%%%%%%%%%%%%%%%%%%%%%%%%%%%%%%%%%%%%%%%%%%%%%
\newpage
\begin{subproblem}\textbf{(b)}
Show that $f:\R\to\R$ is one-to-one if and only if $f(A)\cap f(B) =\varnothing$ for all sets \textit{A,B} with $A\cap B = \varnothing$. 
\end{subproblem}

\begin{solution}\newline
($\Rightarrow$) Assume that \textit{f} is injective, $A,B\in\R\text{ with }A\cap B=\varnothing$.\newline
Take $x\in f(A)\cap f(B)\newline
\implies x\in f(A)\land x\in f(B)\newline
\implies \exists f^{-1}(x)\in A \land \exists f^{-1}(x)\in B\newline
\tab \Rightarrow\Leftarrow\newline
f(A)\cap f(B)=\varnothing$\newline
($\Leftarrow$) Assume that $f(A)\cap f(B)=\varnothing \text{ and } A\cap B=\varnothing\text{ for }A,B\in\R$.\newline
Take $x,y\in\R$. Let $A=\{x\}, B=\{y\}$.\newline
$\implies x\neq y$ with assumption $A\cap B=\varnothing$\newline
$\implies f(x)\neq f(y)$ with assumption $f(A)\cap f(B)=\varnothing$\newline
$\therefore f$ is injective.\flushright$\qedsymbol$
\end{solution}
%%%%%%%%%%%%%%%%%%%%%%%%%%%%%%%%%%%%%%%%%%%%%%%%%%%%%%%%%%%%%%%%%%%%%%%%%%%%
%%%%%%%%%%%%%%%%%%%%%%%%%%%%%%%%%%%%%%%%%%%%%%%%%%%%%%%%%%%%%%%%%%%%%%%%%%%%
\begin{problem}\textbf{A.2.7}
This exercise concerns the notion of preimage. If $f:X\to Y\text{ and } E\subset Y$, then
\begin{center}
    $f^{-1}(E)=\{x:f(x)=y\text{ for some }y\in E\}\subset X$
\end{center}
is called the preimage of \textit{E} under \textit{f}. [There may or may not be an inverse function here; $f^{-1}(E)$ has a meaning even if there is no inverse function].
\end{problem}
%%%%%%%%%%%%%%%%%%%%%%%%%%%%%%%%%%%%%%%%%%%%%%%%%%%%%%%%%%%%%%%%%%%%%%%%%%%%
\begin{subproblem}\textbf{(a)}
Show that $f(f^{-1}(E))\subset E\text{ for every set }E\subset\R$.
\end{subproblem}

\begin{solution}\newline
Take $x\in E$\newline
$f^{-1}(x) \in f^{-1}(E):=\{x:f(x)=y$ for some $y\in E\}\newline
\implies f(f^{-1}(x))\in E\newline
\implies f(f^{-1}(E))\subset E$\flushright$\qedsymbol$
\end{solution}
%%%%%%%%%%%%%%%%%%%%%%%%%%%%%%%%%%%%%%%%%%%%%%%%%%%%%%%%%%%%%%%%%%%%%%%%%%%%
\newpage
\begin{subproblem}\textbf{(b)}
Show that $f^{-1}(f(E))\supset E\text{ for every set }E\subset \R$.
\end{subproblem}

\begin{solution}\newline
Take $x\in E\newline
\implies f(x)\in f(E)\newline
f^{-1}(f(E))=\{a:f(a)=f(X)$ for some $f(x)\in f(E)\newline
x\in f^{-1}(f(E))\newline
\implies E\subset f^{-1}(f(E))$\flushright$\qedsymbol$
\end{solution}
%%%%%%%%%%%%%%%%%%%%%%%%%%%%%%%%%%%%%%%%%%%%%%%%%%%%%%%%%%%%%%%%%%%%%%%%%%%%
\begin{subproblem}\textbf{(c)}
Can you simplify $f^{-1}(A\cup B)\text{ and }f^{-1}(A\cap B)$?
\end{subproblem}

\begin{solution}\newline
Take $x\in f^{-1}(A\cup B)\newline
\iff f(x)\in A\cup B\newline
\iff f(x)\in A \lor f(x)\in B\newline
\iff x\in f^{-1}(A)\lor x\in f^{-1}(B)\newline
\iff x\in f^{-1}(A)\cup f^{-1}(B)\newline
\therefore f^{-1}(A\cup B)=f^{-1}(A)\cup f^{-1}(B)$.\newline\newline
Take $x\in f^{-1}(A\cap B)\newline
\iff f(x)\in A\cap B\newline
\iff f(x)\in A\land f(x)\in B\newline
\iff x\in f^{-1}(A)\land x\in f^{-1}(B)\newline
\iff x\in f^{-1}(A)\cap f^{-1}(B)\newline
\therefore f^{-1}(A\cap B)=f^{-1}(A)\cap f^{-1}(B)$.
\end{solution}
%%%%%%%%%%%%%%%%%%%%%%%%%%%%%%%%%%%%%%%%%%%%%%%%%%%%%%%%%%%%%%%%%%%%%%%%%%%%
\newpage
\begin{subproblem}\textbf{(d)}
Show that $f:\R\to\R$ is one-to-one if and only if $f^{-1}(\{b\})$ contains at most a single point for any $b\in\R$.
\end{subproblem}

\begin{solution}\newline
($\Rightarrow$) Assume that \textit{f} is injective.
Suppose $f^{-1}(\{b\})$ does not contain at most a single point for any $b\in\R$.\newline
Take $x,y\in f^{-1}(\{b\})\newline
\implies f(x)\in \{b\} \land f(y)\in\{b\}$ by definition.\newline
$\Rightarrow\Leftarrow\newline$
($\Leftarrow$) Assume $f^{-1}(\{b\})$ contains at most a single point for any $b\in\R$.\newline
Take $x,y\in\R$. Let $f(x)=f(y)=b.\newline
\implies x=y$ by the assumption that $f^{-1}(\{b\})$ contains at most a single point.\newline
\textit{f} is injective.\flushright$\qedsymbol$
\end{solution}
%%%%%%%%%%%%%%%%%%%%%%%%%%%%%%%%%%%%%%%%%%%%%%%%%%%%%%%%%%%%%%%%%%%%%%%%%%%%
\begin{subproblem}\textbf{(e)}
Show that $f:\R\to\R$ is onto, that is, the range of \textit{f} is all of $\R$ if and only if $f(f^{-1}(E))=E$ for every set $E\subset\R$.
\end{subproblem}

\begin{solution}\newline
($\Rightarrow$) Assume that \textit{f} is surjective.\newline
($\subseteq$) $x\in f(f^{-1}(E))\in E\newline
\implies f(f^{-1}(E))\subseteq E\newline$
($\supseteq$) $x\in E\newline
f^{-1}(x)\in f^{-1}(E):=\{x:f(x)=y\text{ for some } y\in E\}\newline
\implies x\in f(f^{-1}(E))\newline
\implies E\subset f(f^{-1}(E))\newline
\therefore f(f^{-1}(E))=E$.\newline
($\Leftarrow$) Assume that $f(f^{-1}(E))=E$ for every set $E\subset\R\newline$
Suppose \textit{f} is not onto.\newline
$\implies \exists y\in\R$ such that $\forall x\in \R, f(x)\neq y$.\newline
Let $E=\{y\}\newline
\implies f^{-1}(E)=\varnothing\newline
\implies f(f^{-1}(E))=\varnothing\neq\{y\}=E\newline
\Rightarrow\Leftarrow\newline
\therefore f$ is onto.\flushright$\qedsymbol$
\end{solution}
%%%%%%%%%%%%%%%%%%%%%%%%%%%%%%%%%%%%%%%%%%%%%%%%%%%%%%%%%%%%%%%%%%%%%%%%%%%%
\end{document}