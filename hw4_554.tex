\documentclass[12pt]{article}
\usepackage[%
	letterpaper,%
	includeheadfoot,%
	margin=0.5in,%
	headheight=0.21in,%
	footskip=0.51in%
]{geometry}

% Heading information
\newcommand{\documentname}{Homework 4}
\newcommand{\classname}{Math 554 --- H01}
\newcommand{\semester}{Fall 2020}
\newcommand{\lastrevised}{\today}
\newcommand{\duedate}{August 31, 2020}
\newcommand{\yourname}{Andrew Smith}

% standard AMS math packages
\usepackage{amssymb,amsmath,amsthm}

% problem + solution Environments
\usepackage{environ}
\usepackage{tabto}
\usepackage[svgnames]{xcolor}
\usepackage{framed}
\usepackage{changepage}
\definecolor{shadecolor}{rgb}{0.8,1.00,0.95}
% the solutions are printed if the following is 1
% they are not printed if it is 0
\def\printsolutions{1}
% problem environment
\newcounter{ProblemNum}
\NewEnviron{problem}[1][]{%
\refstepcounter{ProblemNum}
\if\printsolutions1
\begin{shaded}
\else
\bigskip
\fi
\noindent
{\bf Problem%\theProblemNum%
}
\if\printsolutions0
\ifx&#1&\else (#1 points) \fi
\fi
\BODY
\if\printsolutions1
\end{shaded}
\fi
}
% subproblem environment
\newcounter{SubProblemNum}[ProblemNum]
\NewEnviron{subproblem}[1][]{%
\refstepcounter{SubProblemNum}
\if\printsolutions1
\begin{shaded}
\fi
\begin{adjustwidth}{2em}{}
\noindent
%\textbf{(\alph{SubProblemNum})}
\if\printsolutions0
\ifx&#1&\else (#1 points) \fi
\fi
\BODY
\end{adjustwidth}
\if\printsolutions1
\end{shaded}
\fi
}
% solution environment
\NewEnviron{solution}{%
\if\printsolutions1
\begin{oframed}
\noindent
\textbf{Solution:}
\BODY
\end{oframed}
\fi
}

% Header and Footer Information
\usepackage{fancyhdr}
\usepackage{lastpage}
\pagestyle{fancy}
\lhead{\classname}
\chead{\documentname \if\printsolutions1 --- \yourname \fi}
\rhead{\semester}
\lfoot{\if\printsolutions1 Due: \duedate \else Last Revised: \lastrevised\fi}
\cfoot{}
\rfoot{\thepage\ of \pageref{LastPage} }
\renewcommand{\headrulewidth}{0.4pt}
\renewcommand{\footrulewidth}{0.4pt}


% standard commands
\newcommand{\N}{\mathbb{N}}
\newcommand{\Z}{\mathbb{Z}}
\newcommand{\Q}{\mathbb{Q}}
\newcommand{\R}{\mathbb{R}}
\newcommand{\C}{\mathbb{C}}

\begin{document}

\if\printsolutions1
\begin{center}
\textbf{\Large \yourname}
\end{center}
\else
\begin{center}
\textbf{\Large Due: \duedate}
\end{center}
\fi

%%%%%%%%%%%%%%%%%%%%%%%%%%%%%%%%%%%%%%%%%%%%%%%%%%%%%%%%%%%%%%%%%%%%%%%%%%%%

\begin{problem}\textbf{1.6.2}
Find sup $E$ and inf $E$ and (where possible) max $E$ and min $E$ for the following examples of sets:
\end{problem}
\begin{subproblem}\textbf{(a)}
$E=\N$
\end{subproblem}
\begin{solution}\newline
$\N$ has no upper bound; therefore, the maximum does not exist and sup $E=\infty$. The greatest lower bound for $N$ is 1, since $1\leq x$ for all $x\in E$. Since $1\in\N$ as well, min \textit{E}=inf \textit{E}=1.
\end{solution}
\begin{subproblem}\textbf{(b)}
$E=\Z$
\end{subproblem}
\begin{solution}\newline
$\Z$ has no upper nor lower bound; therefore, neither the maximum nor minimum exist, and sup $E=\infty$ and inf $E=-\infty$.
\end{solution}
\begin{subproblem}\textbf{(c)}
$E=\Q$
\end{subproblem}
\begin{solution}\newline
$\Q$ has no upper nor lower bound; therefore, neither the maximum nor minimum exist, and sup $E=\infty$ and inf $E=-\infty$.
\end{solution}
\begin{subproblem}\textbf{(d)}
$E=\R$
\end{subproblem}
\begin{solution}\newline
$\R$ has no upper nor lower bound; therefore, neither the maximum nor minimum exist, and sup $E=\infty$ and inf $E=-\infty$.
\end{solution}
\begin{subproblem}\textbf{(e)}
$E=\{-3,2,5,7\}$
\end{subproblem}
\begin{solution}\newline
The greatest lower bound of $E$ is $-3$ since $-3\leqx$ for all $x\in E$. Since $-3\in E$, inf $E$ = min $E$ = $-3$.\newline The least upper bound of $E$ is $7$ since $7\geq x$ for all $x\in E$. Since $7\in E$, sup $E$=max $E$ = $7$.
\end{solution}
\begin{subproblem}\textbf{(f)}
$E=\{x:x^2<2\}$
\end{subproblem}
\begin{solution}\newline
The greatest lower bound of $E$ is $-\sqrt{2}$ since $-\sqrt{2}\leq x$ for all $x\in E$. Since $-\sqrt{2}\notin E$, the minimum does not exist, but inf $E=-\sqrt{2}$.\newline
The least upper bound of $E$ is $\sqrt{2}$ since $\sqrt{2}\geq x$ for all $x\in E$. Since $\sqrt{2}\notin E$, the maximum does not exist, but sup $E=\sqrt{2}$.
\end{solution}
\begin{problem}\textbf{1.6.8}
Let \textit{A} be a set of real numbers and let $B=\{x+r:x\in A\}$ for some number \textit{r}. Find a relation between $sup A$ and $sup B$.
\end{problem}
\begin{solution}
$sup B=sup A+r$\newline
\underline{Proof}\newline
($\geq$) Take $b\in B$.\newline
$b=x+r, x\in A$ for some number $r$.\newline
By definition of supremum, $x+r\leq sup B\implies x\leq supB-r$\newline
So, $sup B-r$ is an upper bound on $A$, since $x\in A$.\newline
So, by definition of supremum, which is the \textit{least} upper bound in particular, $sup A\leq supB-r\newline
\implies supA+r\leq supB$.\newline
($\leq$) By definition, since $x\in A, x\leq sup A$\newline
Now adding $r$ to both sides gives $x+r\leq sup A+r$.\newline
Now $x+r\in B$, so $sup A+r$ is an upper bound on $B$.\newline
So, by definition of supremum, which is the \textit{least} upper bound in particular, $sup B\leq sup A+r$\newline
Together, ($\leq$) and ($\geq$) imply $sup B=sup A+r$\cdot r\flushright$\qedsymbol$
\end{solution}

\begin{problem}\textbf{1.6.9}
Let \textit{A} be a set of real numbers and let $B=\{xr:x\in A\}$ for some positive number \textit{r}. Find a relation between $sup A$ and $sup B$. (What happens if \textit{r} is negative?)
\end{problem}
\begin{solution}
sup $B$ = sup $A\cdot r$. If $r$ is negative, then sup $B$ = inf $A\cdot r$.\newline
\underline{Proof of sup $B$ = sup $A\cdot r$, $r$ positive}\newline
($\geq$)
Take $b\in B$.\newline
$b=x\cdot r$ for some $x\in A$ and some positive number $r$.\newline
Now, $x\cdot r\leq sup B$, by definition of supremum.\newline
Multiplying by $r^{-1}$ on the right, we have $x\leq sup B\cdot r^{-1}$.\newline
Now, since $x\in A$, $sup B\cdot r^{-1}$ is an upper bound for $A$.\newline
Thus, by definition of supremum, which is the \textit{least} upper bound in particular,\newline
$sup A\leq sup B\cdot r^{-1}$.\newline
Multiplying by $r$ on the right, we have $sup A\cdot r\leq sup B$.\newline
($\leq$) Now, take $x\in A$.\newline
By definition, $x\leq sup A$.\newline
Since $r$ is a positive number, we can obtain $x\cdot r\leq sup A\cdot r$.\newline
Since $x\cdot r\in B$, we have that $sup A\cdot r$ is an upper bound on $B$.\newline
Now, by definition of supremum, which is the \textit{least} upper bound in particular, we have\newline
$supB\leq supA\cdot r$\newline
Together, ($\leq$) and ($\geq$) imply sup $B$ = sup $A\cdot r$
\begin{flushright}
$\qedsymbol$
\end{flushright}
\underline{Proof of sup $B$ = inf $A\cdot r, r$ negative }\newline
($\leq$) Take $b\in B$.\newline
$b=x\cdot r$ for some $x\in A$ and some negative number $r$.\newline
Now, $x\cdot r\leq sup B$ by definition of supremum.\newline
Multiplying by $r^{-1}$ on the right, we have $x\geq sup B\cdot r^{-1}$, since $r$ and consequently $r^{-1}$ were negative.\newline
Now, since $x\in A$, $supB\cdot r^{-1}$ is a lower bound for $A$.\newline
Thus, by definition of infrimum, inf $A\geq sup B\cdot r^{-1}$.\newline
Multiplying by $r$ on the right, we have $inf A\cdot r\geq supB$.\newline
($\geq$) Take $x\in A$.\newline

\dots\newline
So $sup B\leq inf A\cdot r\newline
\therefore $sup $B$ = inf $A\cdot r, r$ negative\flushright$\qedsymbol$


\end{solution}

\begin{problem}\textbf{1.6.13}
Let $A$ and $B$ be sets of real numbers and write
\begin{center}
    $C=\{x+y:x\in A,y\in B\}$.
\end{center}
Find a relation among  sup $A$, sup $B$, and sup $C$.

\end{problem}
\begin{solution}
sup $C$ = sup $B$ + sup $A$.\newline
\underline{Proof}\newline
($\leq$) Take $c\in C$.\newline
$c=x+y$ for $x\in A$ and $y\in B$.\newline
Now, $x\leq \text{sup }A$ by definition, and\newline
$y\leq \text{sup }B$ by definition.\newline
Together these imply $x+y\leq \text{sup }A+\text{sup }B$.\newline
Since $\textit{sup }A+\textit{sup }B$ is an upper bound for an arbitrary element in $C$, and, by definition, supremum is the \textit{least} upper bound, we conclude:\newline
sup $C\leq$ sup $A+$ sup $B$\newline
($\geq$) Take $a\in A$.\newline
Now, $\forall b\in B, a+b\leq \text{sup }C$, by definition, since all $a+b\in C$.\newline
$\implies a\leq \text{sup }C-b$, which means that $\textit{sup }C-b$ is an upper bound for the set $A$, since we chose $a$ to be an arbitrary element of that set.\newline
So, since the supremum of a set is the \textit{least} upper bound in particular, we have\newline 
$\text{sup} A\leq\text{sup }C-b$.\newline
$\implies b\leq\textit{sup }C-\textit{sup }A$, which means that $\textit{sup }C-\textit{sup }A$ is an upper bound for the set $B$.\newline
So, since the supremum of a set is the \textit{least} upper bound in particular, we have\newline
$\textit{sup }B\leq\textit{sup }C-\textit{sup }A$\newline
$\implies \textit{sup }A+\textit{sup }B\leq \textit{sup }C$.\newline
Together, ($\leq$) and ($\geq$) imply sup $C$ = sup $B$ + sup $A$\flushright$\qedsymbol$
\end{solution}
%%%%%%%%%%%%%%%%%%%%%%%%%%%%%%%%%%%%%%%%%%%%%%%%%%%%%%%%%%%%%%%%%%%%%%%%%%%%

\end{document}