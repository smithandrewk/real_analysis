\documentclass[12pt]{article}
\usepackage[%
	letterpaper,%
	includeheadfoot,%
	margin=0.5in,%
	headheight=0.21in,%
	footskip=0.51in%
]{geometry}

% Heading information
\newcommand{\documentname}{Homework 9}
\newcommand{\classname}{Math 554 --- H01}
\newcommand{\semester}{Fall 2020}
\newcommand{\lastrevised}{\today}
\newcommand{\duedate}{September 14, 2020}
\newcommand{\yourname}{Andrew Smith}

% standard AMS math packages
\usepackage{amssymb,amsmath,amsthm}

% problem + solution Environments
\usepackage{environ}
\usepackage{tabto}
\usepackage[svgnames]{xcolor}
\usepackage{framed}
\usepackage{changepage}
\definecolor{shadecolor}{rgb}{0.8,1.00,0.95}
% the solutions are printed if the following is 1
% they are not printed if it is 0
\def\printsolutions{1}
% problem environment
\newcounter{ProblemNum}
\NewEnviron{problem}[1][]{%
\refstepcounter{ProblemNum}
\if\printsolutions1
\begin{shaded}
\else
\bigskip
\fi
\noindent
{\bf Problem%\theProblemNum%
}
\if\printsolutions0
\ifx&#1&\else (#1 points) \fi
\fi
\BODY
\if\printsolutions1
\end{shaded}
\fi
}
% subproblem environment
\newcounter{SubProblemNum}[ProblemNum]
\NewEnviron{subproblem}[1][]{%
\refstepcounter{SubProblemNum}
\if\printsolutions1
\begin{shaded}
\fi
\begin{adjustwidth}{2em}{}
\noindent
%\textbf{(\alph{SubProblemNum})}
\if\printsolutions0
\ifx&#1&\else (#1 points) \fi
\fi
\BODY
\end{adjustwidth}
\if\printsolutions1
\end{shaded}
\fi
}
% solution environment
\NewEnviron{solution}{%
\if\printsolutions1
\begin{oframed}
\noindent
\textbf{Solution:}
\BODY
\end{oframed}
\fi
}

% Header and Footer Information
\usepackage{fancyhdr}
\usepackage{lastpage}
\pagestyle{fancy}
\lhead{\classname}
\chead{\documentname \if\printsolutions1 --- \yourname \fi}
\rhead{\semester}
\lfoot{\if\printsolutions1 Due: \duedate \else Last Revised: \lastrevised\fi}
\cfoot{}
\rfoot{\thepage\ of \pageref{LastPage} }
\renewcommand{\headrulewidth}{0.4pt}
\renewcommand{\footrulewidth}{0.4pt}


% standard commands
\newcommand{\N}{\mathbb{N}}
\newcommand{\Z}{\mathbb{Z}}
\newcommand{\Q}{\mathbb{Q}}
\newcommand{\R}{\mathbb{R}}
\newcommand{\C}{\mathbb{C}}

\begin{document}

\if\printsolutions1
\begin{center}
\textbf{\Large \yourname}
\end{center}
\else
\begin{center}
\textbf{\Large Due: \duedate}
\end{center}
\fi

%%%%%%%%%%%%%%%%%%%%%%%%%%%%%%%%%%%%%%%%%%%%%%%%%%%%%%%%%%%%%%%%%%%%%%%%%%%%

\begin{problem}\textbf{2.5.7}
Assume $s_n\neq 0$ for all $n$. Suppose ${\lim_{n \to \infty}s_n=\infty}$, prove that ${\lim_{n \to \infty}\frac{1}{s_n}=0}$. Is the converse true?
\end{problem}
\begin{solution}
Assume that ${\lim_{n \to \infty}s_n=\infty}$.\newline
We want to show that ${\lim_{n \to \infty}\frac{1}{s_n}=\infty}$.\newline
We want to show that for $M\in\R$, there exist $N\in\N$ such that $\frac{1}{s_n}\geq M$ for $n\geq N$.\newline
We know that for $M\in\R$, there exists $N\in\N$ such that $s_n\geq M$ for $n\geq N$.\newline
Let $M=\frac{1}{\varepsilon}$ where $\varepsilon>0$.\newline
Therefore, there exists $N\in\N$ such that $s_n\geq \frac{1}{\varepsilon}$ for all $n\geq N$.\newline
Since $s_n>0$ where $n\geq N$, $s_n\geq \frac{1}{\varepsilon}\implies \varepsilon\geq \frac{1}{s_n}$ and $|\frac{1}{s_n}-0|< \varepsilon$.\newline
We have shown for $\varepsilon>0$, there exists an $N\in\N$ such that $|\frac{1}{s_n}-0|< \varepsilon$ for $n\geq N$.\newline
Therefore, $\lim_{n\to\infty}\frac{1}{s_n}=0$.
\begin{flushright}
$\qedsymbol$
\end{flushright}
For the converse, assume $\lim_{n\to\infty}\frac{1}{s_n}=0$.\newline
Now, by definition, we have for $\varepsilon>0$, there exists an $N\in\N$ such that $|\frac{1}{s_n}|< \varepsilon$ for all $n\geq N$.
We want to show for $M\in\R$, there exists $N\in\N$ such that $s_n\geq M$ for $n\geq N$.\newline
Let $M=\frac{1}{\varepsilon}$ where $\varepsilon>0$.\newline
We have that $|\frac{1}{s_n}|<\varepsilon\implies |s_n|> \frac{1}{\varepsilon}$.\newline
So, we may conclude that $\lim_{n\to\infty}|s_n|=\infty$; however, it is not always the case that $\lim_{n\to\infty}s_n=\infty$ as there may be some negative $s_n$ where $|s_n|\geq \frac{1}{\varepsilon}$, but $s_m\ngeq\frac{1}{\varepsilon}$.
\begin{flushright}
$\qedsymbol$
\end{flushright}
\end{solution}


%%%%%%%%%%%%%%%%%%%%%%%%%%%%%%%%%%%%%%%%%%%%%%%%%%%%%%%%%%%%%%%%%%%%%%%%%%%%
\begin{problem}\textbf{2.6.1}
Which statements are true?
\end{problem}
\begin{subproblem}\textbf{(a)}
If $\{s_n\}$ is unbounded then it is true that either $\lim_{n\to\infty}s_n=\infty$ or else $\lim_{n\to\infty}s_n=-\infty$.
\end{subproblem}
\begin{solution}
Not necessarily. Take $\{0,1,-1,2,-2,...\}$. This sequence is unbounded by the Archimedian Property, and the limit of this sequence does not diverge to positive or negative infinity.
\end{solution}
\begin{subproblem}\textbf{(b)}
If $\{s_n\}$ is unbounded then $\lim_{n\to\infty}|s_n|=\infty$.
\end{subproblem}
\begin{solution}
True. $\{s_n\}$ unbounded means that for $M\in\R$, there exist an $n\in\N$ such that $|s_n|\geq M$.\newline
Thus, $\lim_{n\to\infty}|s_n|=\infty$.
\end{solution}
\begin{subproblem}\textbf{(c)}
If $\{s_n\}$ and $\{t_n\}$ are both bounded then so is $\{s_n+t_n\}$.
\end{subproblem}
\begin{solution}
True. $\{s_n\}$ bounded by $M$ means that all $s_n$ lie in the range $[-M,M]$. Let $M$ be the bound of $s_n$ and let $N$ be the bound of $t_n$. Taking $M+N$ to be the bound of $\{s_n+t_n\}$ guarantees that all $s_n+t_n$ lie in the range $[-(M+N),M+N]$.
\end{solution}
%%%%%%%%%%%%%%%%%%%%%%%%%%%%%%%%%%%%%%%%%%%%%%%%%%%%%%%%%%%%%%%%%%%%%%%%%%%%
\begin{problem}\textbf{2.6.2}
If $\{s_n\}$ is bounded prove that $\{s_n/n\}$ is convergent.
\end{problem}
\begin{solution}
Assume that $\{s_n\}$ is bounded.\newline
We want to show that there is an $L$ such that for $\varepsilon>0$, there exists an $N\in\N$ such that $|s_n/n-L|<\varepsilon$.\newline
We know that $|s_n|\leq M$ for $M\in\R$.\newline
Since $n>0$, we multiply both sides by $\frac{1}{n}$, giving $\frac{|s_n|}{n}\leq \frac{M}{n}<\varepsilon$ for large enough $N$.
\begin{flushright}
$\qedsymbol$
\end{flushright}
\end{solution}

%%%%%%%%%%%%%%%%%%%%%%%%%%%%%%%%%%%%%%%%%%%%%%%%%%%%%%%%%%%%%%%%%%%%%%%%%%%%
\begin{problem}\textbf{2.7.5}
Which statements are true?
\end{problem}
\begin{subproblem}\textbf{(a)}
If $\{s_n\}$ and $\{t_n\}$ are both divergent then so is $\{s_n+t_n\}$.
\end{subproblem}
\begin{solution}
Not necessarily true. Take $s_n=\{n\}$ and $t_n\{-n\}$. Both sequences diverge; however, $\{s_n+t_n\}=\{n-n\}=\{0\}$ converges.
\end{solution}
\begin{subproblem}\textbf{(b)}
If $\{s_n\}$ and $\{t_n\}$ are both divergent then so is $\{s_nt_n\}$.
\end{subproblem}
\begin{solution}
Not necessarily true. Take $s_n=t_n=(-1)^n$. Both $s_n$ and $t_n$ diverge; however, $\{s_nt_n\}=\{1\}$ converges. 
\end{solution}
\begin{subproblem}\textbf{(c)}
If $\{s_n\}$ and $\{s_n+t_n\}$ are both divergent then so is $\{t_n\}$.
\end{subproblem}
\begin{solution}
Not necessarily true. Take $\{s_n\}=(-1)^n$ and $\{t_n\}={0}$. So, $\{t_n\}$ converges while $\{s_n\}$ and $\{s_n+t_n\}$ diverge.
\end{solution}
\begin{subproblem}\textbf{(d)}
If $\{s_n\}$ and $\{s_nt_n\}$ are both divergent then so is $\{t_n\}$.
\end{subproblem}
\begin{solution}
Not necessarily true. Take $\{s_n\}=(-1)^n$ and $\{t_n\}={1}$. So, $\{t_n\}$ converges while $\{s_n\}$ and $\{s_nt_n\}$ diverge.
\end{solution}
\begin{subproblem}\textbf{(e)}
If $\{s_n\}$ is convergent so too is $\{1/s_n\}$.
\end{subproblem}
\begin{solution}
Not necessarily true. Take $\{s_n\}=1/n$ which is convergent; however, $\{1/s_n\}=n$ is divergent.
\end{solution}
\begin{subproblem}\textbf{(f)}
If $\{s_n\}$ is convergent so too is $\{(s_n)^2\}$.
\end{subproblem}
\begin{solution}
True.\newline
We have shown that suppose $\{s_n\},\{t_n\}$ convergent sequences, then $\lim_{n\to\infty}(s_nt_n)=(\lim_{n\to\infty}s_n)(\lim_{n\to\infty}t_n)$.\newline
Therefore, take $\lim_{n\to\infty}(s_ns_n)=(\lim_{n\to\infty}s_n)^2$.\newline
Since $\{s_n\}$ converges, $\{(s_n)^2\}$ also converges.
\end{solution}\begin{subproblem}\textbf{(g)}
If $\{(s_n)^2\}$ is convergent so too is $\{s_n\}$.
\end{subproblem}
\begin{solution}
Not necessarily true. Take $\{s_n\}=(-1)^n$. $s_n$ is divergent while $(s_n)^2$ is convergent.
\end{solution}
%%%%%%%%%%%%%%%%%%%%%%%%%%%%%%%%%%%%%%%%%%%%%%%%%%%%%%%%%%%%%%%%%%%%%%%%%%%%

\end{document}